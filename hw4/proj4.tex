\documentclass{article}
\usepackage{xcolor}
% \usepackage{times}
\usepackage[czech]{babel}
\usepackage[utf8]{inputenc}
\usepackage[IL2]{fontenc}
\usepackage[a4paper,text={17cm,24cm},left=2cm,top=3cm]{geometry}
\usepackage[colorlinks=true,linkcolor=black,urlcolor=black,citecolor=black]{hyperref}
\usepackage{breakurl}
\bibliographystyle{czechiso}

\begin{document}

\begin{titlepage}
    \begin{center}
        \Huge \textsc{Vysoké učení technické v~Brně} \\
        \huge \textsc{Fakulta informačních technologií} \\
        \vspace{\stretch{0.382}}
        \LARGE{Typografie a publikování\,--\,3. projekt} \\
        \Huge {Citace}
        \vspace{\stretch{0.618}}
    \end{center}
    {\Large \today \hfill Josef Kuchař}
    \bigskip
\end{titlepage}

\section{Typografie v~designu}

Typografie neslouží jenom pro psaní textů, ale i pro vyváření grafiky, jak je uvedeno v~\cite{webpage:designhill}.
Typografie v~grafickém designu nemá pevně daná pravidla, ale existují některé zásady, které je vhodné dodržovat.

\subsection{Typografické zásady}
Bez dodržování jednotného stylu v~rámci jedné značky se značka stává nekonzistentní a nekonzistentní značka je nekvalitní.
Této problematice se dopodrobna věnuje~\cite{thesis:capouchova}.

Jednotný grafický styl však nestačí. Je nutné vyvážit velikost písma a velikost mezer, aby bylo čtení pohodlné viz~\cite{webpage:creativebloq}.
Pokud je člověk teprve začínající designer, může využít obecné zásady, které jsou uvedeny v~\cite{webpage:peckadesign}.

Jedna z~nejdůležitějších zásad je kontrast mezi písmem a pozadím. Bez dostatečného kontrastu je celé dílo nečitelné viz~\cite{book:elam}.
To však může být záměr autora viz~\ref{sec:emotions}.

\subsection{Historie typografie v~designu}
Pohled na předchozí zásady se postupně měnil. Pravidla, která dnes bereme jako samozřejmá, mohla být v~minulosti považována za špatná viz~\cite{book:bringhurst}.
V~průběhu historie se samozřejmě měnily i fonty. Vývoj fontů je popsán v~\cite{article:behrens}.

Ani dnes nejsou fonty dokonalé a dokončené. Vývoj fontů se stále rozvíjí a je k~tomu využíváno mnoho moderních technologií.
Problematice změny existujících fontů pomocí počítačových algoritmů se věnuje~\cite{thesis:jiricek}.

\subsection{Předávání emocí}
\label{sec:emotions}
Samotné text sice může předávat emoce, ale bez dodatečného grafického zpracování je to velmi obtížné viz~\cite{article:aynsley}.
Podobnou problematikou se zabývá~\cite{article:kinross}. Uvádí, že věci, které mohou vypadat jako chyba, mohou být záměrně použity pro předání emocí.

\bibliography{proj4}

\end{document}
